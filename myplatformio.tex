\documentclass{article}
\usepackage{amsmath}
\usepackage{tikz}
\usepackage{enumitem}
\begin{document}
\begin{enumerate}
\item A $4 \times 1$ multiplexer with two selector lines is used to realize a Boolean function $F$ having four Boolean variables $X$, $Y$, $Z$, and $W$ as shown below. $S_0$ and $S_1$ denote the least significant bit (LSB) and most significant bit (MSB) of the selector lines of the multiplexer, respectively. $I_0$, $I_1$, $I_2$, $I_3$ are the input lines of the multiplexer.

\begin{tikzpicture}
\draw (0,0) -- (4,0) -- (4,4) -- (0,4) -- cycle;
% Draw the input lines
\draw (0,3.2) -- (-1.0,3.2) node [left] {$Z{W^\prime}$};
\draw (0.3,3.2) node [right] {$I_3$};
\draw (0,2.4) -- (-1.0,2.4) node [left] {$ZW$};
\draw (0.3,2.4) node [right] {$I_2$};
\draw (0,1.6) -- (-1.0,1.6) node [left] {$0$};
\draw (0.3,1.6) node [right] {$I_1$};
\draw (0,0.8) -- (-1.0,0.8) node [left] {${Z^\prime}+W$};
\draw (0.3,0.8) node [right] {$I_0$};
% Draw the selector lines
\draw (1.3,0) -- (1.3,-1.0) node [below] {$X$};
\draw (1.3,0.1) node [above] {$S_1$};
\draw (2.6,0) -- (2.6,-1.0) node [below] {$Y$};
\draw (2.6,0.1) node [above] {$S_0$};
% Draw the output line
\draw [-stealth](4,2.0) -- (5,2.0) node [right] {$F$};
% Draw the multiplexer symbol
\draw (1.7,2.2)  node{$4$};
\draw (2.0,2.2)  node{to};
\draw (2.3,2.2)  node{$1$};
\draw (2.0,1.5)  node{MUX};
\end{tikzpicture}

The canonical sum of product representation of $F$ is:
\begin{enumerate}[label=(\Alph*)]
  \item $F(X, Y, Z, W) = \Sigma m(0,1,3,14,15)$
  \item $F(X, Y, Z, W) = \Sigma m(0,1,3,11,14)$
  \item $F(X, Y, Z, W) = \Sigma m(2,5,9,11,14)$
  \item $F(X, Y, Z, W) = \Sigma m(1,3,7,9,15)$
\end{enumerate}
\end{enumerate}
\end{document}
